 \definecolor{fondTI}{HTML}{869286}


\serie{Puissances}

\begin{exercice}[]
Ecrire chaque expression sous la forme d'une fraction irréductible.\\

\begin{tabular}{p{1,1cm}p{0,8cm}p{1cm}p{0,8cm}p{1cm}p{0,8cm}}
$7^{-2}=$ & & \scriptsize $(-6)^{-1}=$ & & $4^{-3}=$ & \\
& ....... & & ....... & & .......\\
\end{tabular}
\end{exercice}
\medskip
\begin{exercice}[]
Calculer mentalement et exprimer sous la forme d'une fraction lorsque le résultat n'est pas un nombre entier.

\renewcommand{\arraystretch}{1.1}
\begin{tabular}{p{1,1cm}p{0,8cm}p{1cm}p{0,8cm}p{1cm}p{0,8cm}}
$(-5)^2 =$ & ....... & $5^{-2} =$ & ....... & $-5^2 =$ & .......\\
&&&&&\\
$5^3 =$ & ....... & $(-5)^3=$ & ....... & $1^6=$ & .......\\
&&&&&\\
$(-3)^4 =$ & ....... & \scriptsize $-(-4)^2 =$ & ....... & \scriptsize $2014^0 =$ & .......\\
&&&&&\\
$-2^3 =$ & ....... & $(-2)^3 =$ & ....... & $2^{-1} =$ & .......\\
&&&&&\\
\scriptsize $(-2)^{-1} =$ & ....... & $3^{-2} =$ & ....... & \scriptsize $2015^1 =$ & .......\\

\end{tabular}
\end{exercice}
\medskip
\begin{exercice}[]
Compléter chaque égalité à l'aide d'un exposant entier relatif.

\begin{tabular}{p{2cm}p{2,75cm}p{2,5cm}}
$25 = 5^{.....}$ & $-27 = (-3)^{.....}$ & $49 = 7^{.....}$\\
&&\\
$\dfrac{1}{4} = 2^{.....}$ & $\dfrac{5}{7} = \left(\dfrac{7}{5}\right)^{.....}$& $\dfrac{27}{8} = \left(\dfrac{3}{2}\right)^{.....}$\\
\end{tabular}
\end{exercice}
\medskip
\begin{exercice}[]
Ecrire chaque nombre sous la forme $a^n$,

où $a$ et $n$ sont des nombres entiers relatifs.

(a) 32 \hspace{0.5cm} (b) 27 \hspace{0.5cm} (c) 100 000 \hspace{0.5cm} (d) $\dfrac{1}{81}$
\end{exercice}
\medskip
\begin{exercice}[]
Calculer chaque expression en détaillant les étapes.

\begin{tabular}{m{3.8cm}m{3.5cm}}
A = $2 + 2^3 \times 5^2$ & B = $-3^2 + 8 : 2^2$\\
&\\
C = $2^2 + 2^{-2}$ & D = $2^2 \times 2^{-2}$\\
&
\end{tabular}
\end{exercice}
\medskip
\begin{exercice}[]
Ecrire chaque expression sous la forme $a^n$,

où $n$ est un nombre entier relatif.

\renewcommand{\arraystretch}{0.8}
\begin{tabular}{p{4cm}p{4cm}}
(a) $3^2 \times 3^5$ & (b) $(-6)^{10} \times (-6)^7$\\
&\\
(c) $\left(\dfrac{5}{3}\right)^4 \times \left(\dfrac{5}{3}\right)^3$ & (d) $\left(\dfrac{4}{7}\right)^8 \times \dfrac{4}{7}$\\
&\\

(e) $\left(\dfrac{1}{3}\right)^{12} \times \left(\dfrac{1}{3}\right)^{-7}$ & (f) $(-9)^{-8} \times (-9)$\\
&\\
(g) $\dfrac{2^6}{2^4}$ & (h) $\dfrac{5^{-3}}{5^7}$\\
&\\

(i) $\left(2^3\right)^4$ & (j) $\left(1,2^5\right)^{-2}$\\
&\\
(k) $\left(\left(\dfrac{5}{6}\right)^{-1}\right)^4$ & (l) $\left(\left(-2\right)^{-6}\right)^{-7}$\\
&\\
(m) $9^2 \times 7^2$ & (n) $(-11)^{-1} \times 3^{-1}$\\
&\\
(o) $(-5)^{-6} \times (-7)^{-6}$ & (p) $\left(\dfrac{2}{3}\right)^5 \times \left(\dfrac{9}{10}\right)^5$\\
&\\
(q) $\dfrac{24^7}{3^7}$ & (r) $\dfrac{(-4,5)^{-3}}{9^{-3}}$\\
\end{tabular}
\end{exercice}
\medskip
\begin{exercice}[]
Pour chacune des écritures suivantes :
\begin{itemize}
\item dire s'il s'agit ou non d'une écriture scientifique ;
\item donner son écriture décimale.
\end{itemize}

\renewcommand{\arraystretch}{0.8}
\begin{tabular}{p{4cm}p{4cm}}
(a) $5 \times 10^4$ & (b) $21,8 \times 10^6$\\
&\\
(c) $6,5 \times 10^{-2}$ & (d) $0,4 \times 10^{-5}$\\
&\\
(e) $-3 000 \times 10^{-6}$ & (f) $-1,7 \times 10^{-1}$\\
\end{tabular}
\end{exercice}
\medskip
\begin{exercice}[]
Donner l'écriture scientifique de chacun des nombres suivants.

\renewcommand{\arraystretch}{0.8}
\begin{tabular}{p{4cm}p{4cm}}
(a) $471$ & (b) $-360~000$\\
&\\
(c) $0,025$ & (d) $1234,567$\\
&\\
(e) $920~000~000$ & (f) $-0,000~000~000~03$\\
\end{tabular}
\end{exercice}
\medskip
\begin{exercice}[]
Calculer les expressions suivantes et donner l'écriture scientique du résultat.
\begin{description}
\item $A=\dfrac{90 \times 10^7 \times 1,4 \times10^{-2}}{12 \times (10^3)^5}$\\
\item $B=\dfrac{2400 \times 10^{-3} \times 0,28 \times10^{-2}}{8,4 \times (10^{-10})^2}$
\end{description}
\end{exercice}

\newpage

\serie{Développer et factoriser}

\begin{exercice}[]
Développer et réduire les expressions algébriques suivantes :

\begin{tabular}{m{4.75cm}m{2.75cm}}
{A$_1=8 \times (-6x-9)+1$} \\
&\\
{A$_2=6x+5 \times (-5x-2)$}\\
&\\
{A$_3=10 \times (x-4)+4x+3$}\\
\end{tabular}
\end{exercice}

\begin{exercice}[]
Développer et réduire les produits suivants :

\begin{tabular}{m{5.75cm}m{1.75cm}}
B$_1=(x+2) \times (x+5)$ \\
&\\
B_2 {$=\left(x+\dfrac{5}{4} \right) \left (3x-\dfrac{1}{2} \right)$}
&\\
B$_3=(2x-3)x+(x+5)(3x-4)$\\
&\\
B$_4=(6-a)(2a+5)-(-a+1)(3a-10)$}\\
\end{tabular}
\end{exercice}

\begin{exercice}[]
Factoriser chacune des expressions suivantes, en identifiant au préalable un facteur commun :

\footnotesize \hspace{0.5cm}
\begin{tabular}{m{3.75cm}m{3.75cm}}
C$_1=3x-9$ & C$_2=6,3y+6,3$\\
&\\
C$_3=5x-12x^2$ & C$_4=60a^2-120a$\\
&\\
C$_5=b^4+3b^2$ & C$_6=48x^3+16x^2-32x$\\\\
\end{tabular}
\end{exercice}

\begin{exercice}[]
Développer et réduire les produits suivants, en utilisant une des trois premières identités remarquables :

\vspace{0.25cm}
\begin{tabular}{m{3.75cm}m{3.75cm}}
D$_1=(x+2)^2$ & D$_2=(a-5)^2$\\
&\\
D$_3=\left(x+\dfrac{2}{3} \right)^2$ & D$_4=(7-5x)(7+5x)$\D$_1=(x+2)^2$ & D$_2=(a-5)^2$\\
&\\
D$_5=\left(5x-10 \right)^2$ & D$_6=(5x-10)(5x+10)$\\
&\\
D$_7=(5x-2)^2$ & D$_8=(3x-4)(4+3x)$\\
&\\
D$_9=\left(\dfrac{4}{7}x- \dfrac{4}{3} \right)^2$ & D$_10=-(9x-5)(5-9x)$\\
\end{tabular}
\end{exercice}

\begin{exercice}[]
Factoriser, lorsque c'est possible, chacune des expressions suivantes, en utilisant une des trois premières identités remarquables :

\vspace{0.25cm}
\begin{tabular}{m{3.75cm}m{3.75cm}}
E$_1=x^2+6x+9$ & E$_2=a^2-10a+25$\\
&\\
E$_3=x^2-81$ & E$_4=z^2+64$\\
&\\
E$_5=121-4z^2$ & E$_6=25x^2-10x+4$\\
&\\
E$_7=5t^2-2\sqrt{5}t+1$ & E$_8=3n^2-48$\\
\end{tabular}
\end{exercice}

\begin{exercice}[]
Développer et réduire les produits suivants, en utilisant l'identité remarquable appropriée :

\vspace{0.25cm}
\begin{tabular}{m{3.75cm}m{3.75cm}}
F$_1=3(4z-1)^2$ & F$_{2}=-2(9+4a)^2$\\
&\\
F$_3=(4x-3y)(4x-3y)$ & F$_{4}=(-2x^2+3y)^2$\\
&\\
F$_{5}=(y-5)(y+9)$ & {\scriptsize F$_{6}=\left(\dfrac{1}{2}-x\right)\left(x+\dfrac{1}{2}\right)$}\\
\end{tabular}
\end{exercice}

\begin{exercice}[]
Factoriser chacune des expressions suivantes en utilisant la quatrième identité remarquable :

\begin{tabular}{m{3.75cm}m{3.75cm}}
G$_1=x^2+5x+6$ & G$_2=t^2+12t+27$\\
&\\
G$_3=z^2-3z-28$ & G$_4=x^2+{4}x+32$\\
\end{tabular}
\end{exercice}

\begin{exercice}[]
Factoriser chacune des expressions suivantes en utilisant une identité remarquable :

\vspace{0.25cm}
\begin{tabular}{m{3.75cm}m{3.75cm}}
H$_1=16x^2-36$ & H$_2=t^2-7t+10$\\
&\\
H$_3=a^4b^2-a^6$ & H$_4=-y^2+{7}y+8$\\
&\\
H$_5=4z^2+y^2+4zy$ & H$_6=\dfrac{1}{4}x^4-9y^2$\\\\
\end{tabular}
\end{exercice}

\begin{exercice}[]
Factoriser chacune des expressions suivantes :

\vspace{0.25cm}
\begin{tabular}{m{3.75cm}m{3.75cm}}
I$_1=-81x^2+(x-6)2$ & I$_2=4x^2-32x+64$\\
&\\
I$_3=(5x+2)(5x-8)+(-5x-1)(5x+2)$ & I$_4=64x^2-64$\\
&\\
I$_5=(4x+3)(9x-10)+9x-10$ & I$_6=-(3x+6)(9x-3)+(3x+6)^2$\\\\
\end{tabular}
\end{exercice}

\serie{Divers}

\begin{exercice}[]
La vistesse de la lumière est d'environ} $3 \times 10^5$ km/s.

\begin{enumerate}
\item La lumière met un septante-cinquième de seconde pour aller d'un satellite à la Terre.

Calculer la distance séparant ce satellite de la Terre.

\item La lumière émise par le Soleil met environ $8$ minutes et $30$ secondes pour parvenir sur Terre.

Calculer la distance entre la Terre et le Soleil.

On exprimera la réponse sous forme scientifique.
\end{enumerate}
\end{exercice}

\begin{exercice}[]
Un clou en fer a une masse de $11,69$ g.

\vspace{0,15cm} La masse de l'atome de fer est de} $9,352 \times 10^{-26}$ kg.

Calculer le nombre d'atomes de fer contenus dans ce clou.
\end{exercice}

\begin{exercice}[]
\begin{enumerate}
\item Développer et réduire l'expression littérale :

E $=(a-1)^2+a^2+(a+1)^2$

\item En déduire trois nombres entiers consécutifs dont la somme des carrés est égale à $4~802$.
\end{enumerate}
\end{exercice}

