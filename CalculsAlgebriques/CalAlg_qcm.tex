

%%%%%%%%%%%%%%%%%%%%%%%%%%%%%%%%%%%%%%%%%%%%%%%%%%%%%%%%%%%%%%%%%%%%%%%%%%%

\QCMautoevaluation{Pour chaque question, plusieurs réponses sont proposées. Déterminer celles qui sont correctes.} 

\begin{QCM}
  \begin{GroupeQCM} 
    \begin{exercice}
      $\dfrac{6\times 10^3 \times 28 \times 10^{-2}}{14 \times 10^{-3}}$ est égal à:
      \begin{ChoixQCM}{4}
      \item $12 \times 10^{-9}$
      \item $12 \times 10^{4}$
      \item $0,12$
      \item $1,2 \times 10^{3}$
      \end{ChoixQCM}
\begin{corrige}
     \reponseQCM{b} 
   \end{corrige}
    \end{exercice}

    \begin{exercice}
      $(8^{-2})^8$ est
      \begin{ChoixQCM}{4}
      \item égal à $8^{-10}$
      \item une puissance de 4
      \item égal à $-16^8$
      \item égal à $\left( \dfrac{1}{8^2}\right)^8$
      \end{ChoixQCM}
      \begin{corrige}
     \reponseQCM{d}
   \end{corrige}
    \end{exercice}

    
    \begin{exercice}
      L'écriture scientifique de 0,005678 est :
      \begin{ChoixQCM}{4}
      \item $5,678\cdot 10^{-4}$
      \item $5,678\cdot 10^{-3}$
      \item $5,678\cdot 10^{-5}$
      \item $0,5678\cdot 10^{-2}$
      \end{ChoixQCM}
      \begin{corrige}
     \reponseQCM{b}
   \end{corrige}
    \end{exercice}


    \begin{exercice}
      $3(x+1)-(x+1)(x-2)$ est égal à :
      \begin{ChoixQCM}{4}
      \item $(x+1)(x-5)$
      \item $(x+1)(-x+1)$
      \item $-x^2+2x-1$
      \item $-x^2+4x+5$
      \end{ChoixQCM}
      \begin{corrige}
     \reponseQCM{d}
   \end{corrige}
    \end{exercice}


    \begin{exercice}
      $9a^2-4=...$
      \begin{ChoixQCM}{4}
      \item $(3a-2)^2$
      \item $(3a-2)(3a+2)$
      \item $5a^2$
      \item $(9a-4)(9a+4)$
      \end{ChoixQCM}
      \begin{corrige}
     \reponseQCM{b}
   \end{corrige}
    \end{exercice}
    
    
     \begin{exercice}
      $x^2-5x+6=...$
      \begin{ChoixQCM}{4}
      \item $(x-6)(x+1)$
      \item $(x+6)(x+1)$
      \item $(x-2)(x-3)$
      \item $(x-2)(x+3)$
      \end{ChoixQCM}
      \begin{corrige}
     \reponseQCM{b}
   \end{corrige}
    \end{exercice}
    
     \begin{exercice}
      $\left( \dfrac{2}{3}a+1\right) \left( 1-\dfrac{2}{3}a\right) =...$
      \begin{ChoixQCM}{4}
      \item $\dfrac{4}{6}a^2-1$
      \item $1-\dfrac{4}{9}a^2$
      \item $\dfrac{4}{9}a^2-1$
      \item $\dfrac{4}{9}a^2+1$
      \end{ChoixQCM}
      \begin{corrige}
     \reponseQCM{b}
   \end{corrige}
    \end{exercice}
    
    
     \begin{exercice}
      $B=25x^2-15x+9$
      \begin{ChoixQCM}{4}
      \item On ne peut pas factoriser $B$
      \item $B=(5x-3)^2$
      \item $B=(5x+3)^2$
      \item $B=(5x-3)^2+15x$
      \end{ChoixQCM}
      \begin{corrige}
     \reponseQCM{ad}
   \end{corrige}
    \end{exercice}
 \end{GroupeQCM}  
\end{QCM}  
    

    
   

  
