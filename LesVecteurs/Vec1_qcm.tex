

\QCMautoevaluation{Pour chaque question, plusieurs réponses sont
  proposées.  Déterminer celles qui sont correctes.}


\begin{QCM}

\begin{GroupeQCM}

\begin{center}
\begin{tikzpicture}[scale=0.8][general]
 \draw (0,0)--(1,3)--(3,3)--(2,0)--cycle;
 \draw (1,3)--(3,0)--(5,-0.3)--(3,3);
 \draw (0,0)node[left] {$F$};
 \draw (1,3)node[above] {$A$};
 \draw (3,3)node[above] {$B$};
 \draw (2,0)node[below] {$C$};
 \draw (5,-0.3)node[below] {$D$};
 \draw (3,0)node[below] {$E$};
 \draw[color=C1] (0.5,1.5)node {{\boldmath $\infty$}};
 \draw[color=C1] (2.5,1.5)node {{\boldmath $\infty$}};
 \draw[color=F1] (2,3)node[rotate=90] {{\boldmath $\approx$}};
 \draw[color=F1] (1,0)node[rotate=90] {{\boldmath $\approx$}};
 \draw[color=F1] (4,-0.1)node[rotate=90] {{\boldmath $\approx$}};
\end{tikzpicture}
\end{center}

\begin{exercice}L'image de $F$ dans la translation de vecteur $\vv{AB}$ est le point:
\begin{ChoixQCM}{2}
\item $C$
\item $E$
\end{ChoixQCM}
\begin{corrige}
\reponseQCM{a}
\end{corrige}
\end{exercice}

\begin{exercice}$\vv{AB}+\vv{BD}=\vv{AD}$
\begin{ChoixQCM}{2}
\item vrai
\item faux
\end{ChoixQCM}
\begin{corrige}
\reponseQCM{a}
\end{corrige}
\end{exercice}

\begin{exercice}$AB+BD=AD$
\begin{ChoixQCM}{2}
\item vrai
\item faux
\end{ChoixQCM}
\begin{corrige}
\reponseQCM{b}
\end{corrige}
\end{exercice}

\begin{exercice}$ABDE$ est un parallélogramme.
\begin{ChoixQCM}{2}
\item vrai
\item faux
\end{ChoixQCM}
\begin{corrige}
\reponseQCM{b}
\end{corrige}
\end{exercice}

\begin{exercice}$FCBA$ est un parallélogramme.
\begin{ChoixQCM}{2}
\item vrai
\item faux
\end{ChoixQCM}
\begin{corrige}
\reponseQCM{a}
\end{corrige}
\end{exercice}


\begin{exercice}$\vv{AB}=\vv{CF}$
\begin{ChoixQCM}{2}
\item vrai
\item faux
\end{ChoixQCM}
\begin{corrige}
\reponseQCM{b}
\end{corrige}
\end{exercice}

\begin{exercice}$\vv{DE}=\vv{BA}$
\begin{ChoixQCM}{2}
\item vrai
\item faux
\end{ChoixQCM}
\begin{corrige}
\reponseQCM{b}
\end{corrige}
\end{exercice}

\begin{exercice}$DE=BA$
\begin{ChoixQCM}{2}
\item vrai
\item faux
\end{ChoixQCM}
\begin{corrige}
\reponseQCM{a}
\end{corrige}
\end{exercice}

\begin{exercice}$\vv{AB}+\vv{AF}=\vv{AC}$
\begin{ChoixQCM}{2}
\item vrai
\item faux
\end{ChoixQCM}
\begin{corrige}
\reponseQCM{a}
\end{corrige}
\end{exercice}

\begin{exercice}$\vv{CB}+\vv{AB}=\vv{CA}$
\begin{ChoixQCM}{2}
\item vrai
\item faux
\end{ChoixQCM}
\begin{corrige}
\reponseQCM{b}
\end{corrige}
\end{exercice}



\end{GroupeQCM}
\end{QCM}

  
