% a priori, après l'impression du livre, tout ce fichier devrait être
% fusionné avec la classe principale, dernière version

\AtBeginDocument{\color{Noir}}

\renewcommand*\StringPrerequis{Connaissances
  n\'ecessaires \`a ce chapitre}

\usepackage{textcomp,pgfplots}
% standalone ne gère pas graphicspath... hum...
\newcommand{\standalonepath}[1]{#1}
% Au début de chaque chapitre on redéfinira cette commande en
% indiquant le chemin

\frenchbsetup{og=«,fg=»,}

\renewcommand\PrefixeCorrection{Corrections/}
\DeclareRemLike{intuition}{Idée intuitive}
\DeclareRemLike{exemples}{Exemples}


\newcommand*\StringExemples{Exemples}
\makeatletter
\newcommand*\smc@cartoucheexemples{% au pluriel
  \begin{pspicture}(-\ExempleVRuleWidthFrame,0)
                 (\ExempleWidthFrame,\ExempleHeightFrame)
    \psframe*[linewidth=0pt,linecolor=ExempleEdgeFrameColor]
             (-\ExempleVRuleWidthFrame,-\ExempleHRuleWidthFrame)
             (\ExempleWidthFrame,\ExempleHeightFrame)
    \psframe*[linewidth=0pt,linecolor=ExempleBkgFrameColor]
             (0mm,-0mm)(\ExempleWidthFrame,\ExempleHeightFrame)
    \rput[B](\dimexpr\ExempleWidthFrame/2,0){%
      \ExempleTitleFont
      \textcolor{ExempleTitleColor}{\StringExemples}%
    }
  \end{pspicture}%
}

\newenvironment{exemples*1}[1][]{%
  \par\addvspace{\BeforeExempleVSpace}
  \let\correction\smc@one@exemplecorrection
  \let\itemize\smc@exempleitemize
  \let\enditemize\endsmc@exempleitemize
  \let\colitemize\smc@exemplecolitemize
  \let\endcolitemize\endsmc@exemplecolitemize
  \let\enumerate\smc@exempleenumerate
  \let\endenumerate\endsmc@exempleenumerate
  \let\colenumerate\smc@exemplecolenumerate
  \let\endcolenumerate\endsmc@exemplecolenumerate
  \let\partie\smc@nopartie
  \let\exercice\smc@noexercice
  \let\endexercice\endsmc@noexercice
  \let\corrige\smc@nocorrige
  \let\endcorrige\endsmc@nocorrige
  \def\smc@currpart{Exemple}%
  \hspace*{\dimexpr \SquareWidth*3}%
  \color{ExempleRuleColor}%
  \vrule width \RuleWidth
  \hspace*{\dimexpr \SquareWidth-\RuleWidth}%
  \minipage[t]{\dimexpr\linewidth-\SquareWidth*4-\ExtraMarginRight}
    \smc@cartoucheexemples
    \space
    \color{Noir}%
    \ignorespaces
}
{%
  \endminipage
  \par
}

\DeclareRemLike{consequence}{Conséquence}
\DeclareRemLike{rappel}{Rappel}
\DeclareRemLike{valeurspart}{Valeurs particulières}
%\DeclareRemLike{theoreme}{Théorème}
% \NewThema{SP}
%          {sp}
%          {stat. et probabilités}
%          {Stat. et probabilités}
%          {STATISTIQUES\\ PROBABILITÉS}
%          {PartieStatistique}
%          {PartieStatistique}

\NewThema{A}
         {a}
         {analyse}
         {Analyse}
         {ANALYSE}
         {PartieFonction}
         {A3}


% Attention, il faudra ajouter un renewcommand \ListeMethodesThemes
% pour afficher la liste des méthodes à la fin du livre

% Si ils veulent changer la couleur du numéro des exos des
% auto-évaluations il faudra aller voir
% \colorlet{CorrigeNumExerciceFrameBkg}{J1}
 
\newcommand{\N}{\mathbb{N}} 
\newcommand{\R}{\mathbb{R}}
\newcommand{\Z}{\mathbb{Z}}
\renewcommand{\cfrac}[2]{{\displaystyle\frac{%
  \vrule height10pt depth0pt width0pt #1}{#2}}%
  \kern-\nulldelimiterspace}
%\usepackage{casio-fx,ti83symbols}

\newcommand\renvoimethode[1]{%
  Méthode \ref{#1}, p.~\pageref{#1}%
}

\newcommand*\calculatrice{%
  \psframebox[framesep=1pt,linewidth=\LogoLineWidth,
              linecolor=TiceLineColor, fillstyle=solid,
              fillcolor=TiceBkgColor, framearc=0.6]{%
    \TiceFont
    \textcolor{TiceTextColor}{CALC}%
  }
}
\newcommand\calc{\calculatrice}

% Chapitre G1 et G3
\newcommand{\covec}[2]{\left(\begin{array}{c} #1\\#2\end{array}\right)}

% Chapitre G2
\usepackage{tipa}
\newcommand{\arc}[1]{%
  \setbox9=\hbox{#1}%
  \ooalign{\resizebox{\wd9}{\height}{\texttoptiebar{\phantom{A}}}\cr#1}}

% Chapitre A4


\DeclareMathOperator{\e}{e}
\renewcommand{\cosh}{\operatorname{ch}}
\renewcommand{\sinh}{\operatorname{sh}}
\renewcommand{\tanh}{\operatorname{th}}

\newcommand*{\StringLEMM}{LEMME\footnote{Un \MotDefinition{lemme}{} est un résultat préliminaire ou  intermédiaire qui intervient parfois dans la preuve d'un théorème lorsqu'elle est un peu longue.}}
\newcommand*{\StringLEMME}{LEMME}
\DeclareDefLike{lemme}{\StringLEMME}
\DeclareDefLike{lemm}{\StringLEMM}

\newcommand*{\StringPROPRIETA}{PROPRIÉTÉ (admise)}
\DeclareDefLike{proprieta}{\StringPROPRIETA}

\usepackage{wrapfig}

% Environnement général pour toutes les fiches
\newcommand\AnnexeTICE{%
  \ChangeAnnexe{C2}{A1}{G1}{Blanc}%
  \annexe{}%
}
% Déclaration de l'environnement pour une fiche
\DeclareTPLike{ficheTICE}{Fiche}
              {TPTopColor}
              {TPBottomColor}
              {TPTitleColor} 
% Définition d'une commande \souspartie pour les besoins de la fiche
% TICE. C'est la commande \partie un peu revue. Je crée les mêmes
% paramètres de contrôle que pour TPPartie en mettant TPSousPartie à
% la place.

\colorlet{TPSousPartieColor}{J1}
\colorlet{TPSousPartieBkgColor}{C2}
\colorlet{TPSousPartieNumColor}{Blanc}
\newcommand*\TPSousPartieFont{\fontsize{10}{12}\sffamily\bfseries}
\def\BeforeTPSousPartieVSpace{3mm plus1mm minus1mm}
\def\AfterTPSousPartieVSpace{0mm plus1mm}
\edef\TPSousPartieHSep{\the\dimexpr\ItemRuleWidth+1.5mm}

\newcommand*\souspartie[1]{%
  \colorlet{smc@curr@partiecolor}{TPSousPartieNumColor}%
  \colorlet{smc@curr@partiebkgcolor}{TPSousPartieBkgColor}%
  \let\smc@curr@partiefont\TPSousPartieFont
  \par\addvspace{\BeforeTPSousPartieVSpace}
  \leavevmode
  \psframe*[linecolor=smc@curr@partiebkgcolor]
           (0,\ItemRuleDepth)(\ItemRuleWidth,\ItemRuleHeight)
  \hspace*{\TPSousPartieHSep}%
  \textcolor{TPSousPartieColor}{\TPSousPartieFont #1}
  \par\nobreak\addvspace{\AfterTPSousPartieVSpace}
}%

\newcommand\RoseItalTice[1]{\emph{\textcolor{C2}{#1}}}

%%% La présentation d'un texte en vis à vis d'une image n'est pas tout
%%% à fait un habillage et la répétition de tels éléments risque de
%%% poser des problèmes. Il vaut mieux se faire son propre « habillage
%%% »
\newcommand\ImageDroite[2]{%
  % #1 = texte
  % #2 = image (ou autre)
  \setbox4=\hbox{#2}%
  \dimen4=\dimexpr\ht4+\dp4-0.7\baselineskip
  \par
  \begin{tabularx}{\linewidth}{@{}Xc@{}}
    #1 & \raisebox{-\dimen4}{#2} %\box4
  \end{tabularx}
  \par
}
\newcommand{\commandetice}[1]{%
  \bgroup
  \shorthandoff{;:!?}%
  \texttt{#1}%
  \egroup
}

\newcommand\touchecalc[1]{%
  \tikz[baseline=-0.5ex]{\node at (0,0) [rounded corners = 2pt, draw, line width=.25pt]
    {\footnotesize\textsf{#1}}}%
}

\DeclareFontFamily{U}{tipa}{}
\DeclareFontShape{U}{tipa}{m}{n}{<->tipa10}{}
\newcommand{\arc@char}{{\usefont{U}{tipa}{m}{n}\symbol{62}}}%

\newcommand{\overarc}[1]{\mathpalette\arc@arc{#1}}

\newcommand{\arc@arc}[2]{%
  \sbox0{$\m@th#1#2$}%
  \vbox{
    \hbox{\resizebox{\wd0}{\height}{\arc@char}}
    \nointerlineskip
    \box0
  }%
}

% Pour A2

\def\psThomae{\pst@object{psThomae}}
\def\psThomae@i(#1,#2){%
   \begin@ClosedObj
   \addto@pscode{
     \psk@dotsize
     1 1 500 {
       dup
       /ipSave ED
       /ip ED
       1 1 500 {
         dup
         /iqSave ED
         /iq ED
         {
           iq 0 le { exit } if
           ip iq mod
           /ip iq def
           /iq ED
         } loop
         ip 1 eq {
           \psk@dotsize
           \@nameuse{psds@\psk@dotstyle}
           \pst@usecolor\pslinecolor ipSave iqSave div 1 iqSave div 
\tx@ScreenCoor
           2 copy moveto Dot
         } if
       } for
     } for
   }%
   \end@ClosedObj%
}

\renewcommand\smc@AfficheListeMethodesTheme[2]{%
  \expandafter\ifx\csname ifsmc@lom#1\endcsname\iftrue
    \csname smc@thema#2Color\endcsname
    \expandafter\smc@bandeaulistemethodes
      \expandafter{\csname StringListeMethode#2\endcsname}
    \ifnum \smc@NombreColonnesListeMethodes=\@ne
      \@starttoc{lom#1}
    \else
      \begin{multicols}{\smc@NombreColonnesListeMethodes}
        \@starttoc{lom#1}
      \end{multicols}
    \fi
  \fi
  \newpage
}
\fancypagestyle{empty}{%
  \fancyhead{}
  \fancyfoot{}
} 

\renewcommand*\AfficheCorriges[1][\NombreColonnesCorriges]{%
  \clearpage
  \label{toutes-solutions}
  \pagestyle{corrige}
  \thispagestyle{firstcorrige}
  \rput[Bl](0,9mm){\CorrigeTitleFont \MakeUppercase{\StringCorriges}}
  \vspace*{-5mm}
  \begingroup
  \columnsep \dimexpr \SquareWidth*2
  \columnseprule \CorrigeRuleWidth
  \def\columnseprulecolor{\color{ExerciceColumnRuleColor}}%
  \xdef\smc@NbColonneCorrige{#1}%
  \begin{multicols*}{#1}
  \raggedcolumns
    \@starttoc{cor}%
  \end{multicols*}
  \endgroup
}


\renewcommand\smc@preinsertlexiquefinal[3][]{%
  \@ifmtarg{#1}%
    {\smc@sansdiacritique{#2}}%
    {\smc@sansdiacritique{#1}}%
  \ifcsname affiche-\smc@tri\endcsname
  \else
    \expandafter\gdef\csname affiche-\smc@tri\endcsname{true}%
    \@ifmtarg{#1}%
      {\smc@sansdiacritique{#2}}%
      {\smc@sansdiacritique{#1}}%
    \global\advance\smc@numlexique \@ne
    \@ifmtarg{#1}%
      {\smc@@preFirstUppercase#2\@nil#3\@nil}%
      {\expandafter\protected@xdef\csname
lexique\the\smc@numlexique\endcsname
        {%
          \protect\textcolor{LexiqueEntreeColor}{%
            \protect\LexiqueEntreeFont #2%
          }%
%%%       \space\hbox to4.4em{\rdotfill}\kern0em\penalty0
          ~%%%
          \hspace*{\LexiquePageWidth}\penalty0
          \hspace{-\LexiquePageWidth}\dotfill
          \ifnum\csname nb-\smc@tri\endcsname>\@ne
            \protect\emph{ Pages~\csname pages-\smc@tri\endcsname}%
          \else
            \protect\emph{ Page~\csname pages-\smc@tri\endcsname}%
          \fi
        }%
      }%
    \expandafter\xdef\csname tri\the\smc@numlexique\endcsname
      {\smc@tri}%
  \fi
}
\long\def\smc@@preFirstUppercase#1#2#3\@nil#4\@nil{%
  \def\smc@arg{#1}%
  \ifx\smc@arg\smc@IeC
    \expandafter\protected@xdef\csname lexique\the\smc@numlexique\endcsname
      {%
        \protect\textcolor{LexiqueEntreeColor}
        {%
          \protect\LexiqueEntreeFont
          \MakeUppercase{#1#2}%
          \MakeLowercase{#3}%
        }%
        %%% \space\hbox to4.4em{\rdotfill}\kern-0.44em\penalty0
        ~%%%
        \hspace*{\LexiquePageWidth}\penalty0
        \hspace{-\LexiquePageWidth}\rdotfill
        \ifnum\csname nb-\smc@tri\endcsname>\@ne
          \protect\emph{ Pages~\csname pages-\smc@tri\endcsname}%
        \else
          \protect\emph{ Page~\csname pages-\smc@tri\endcsname}%
        \fi
      }%
  \else
    \expandafter\protected@xdef\csname lexique\the\smc@numlexique\endcsname
      {%
        \protect\textcolor{LexiqueEntreeColor}
        {%
          \protect\LexiqueEntreeFont
          \MakeUppercase{#1}%
          \MakeLowercase{#2#3}%
        }%
%%%     \space\hbox to4.4em{\rdotfill}\kern-0.44em\penalty0
        ~%%%
        \hspace*{\LexiquePageWidth}\penalty0
        \hspace{-\LexiquePageWidth}\rdotfill
        \ifnum\csname nb-\smc@tri\endcsname>\@ne
          \protect\emph{ Pages~\csname pages-\smc@tri\endcsname}%
        \else
          \protect\emph{ Page~\csname pages-\smc@tri\endcsname}%
        \fi
      }%
  \fi
} 

\makeatother

%\input{G1/biton}